\chapter{CONCLUSÕES E TRABALHOS FUTUROS} \label{conclusao}

Esta tese contribui com o desenvolvimento de algoritmos para geração de múltiplas árvores 
circulatórias dentro de um domínio de perfusão. Um dos algoritmos elaborados gera uma floresta de árvores 
circulatórias controlando a invasão de cada árvore no território, ou seja, domínio de perfusão da outra. O outro algoritmo construído usa uma adaptação do diagrama de Voronoi
para subdividir o domínio de perfusão em subdomínios disjuntos e, posteriormente,  gera
dentro destes subdomínios cada árvore da floresta.

Os resultados  oriundos dos modelos de árvores circulatórias gerados pelos algoritmos propostos foram alcançados em diferentes domínios de perfusão (bidimensionais  
e tridimensionais) e demonstram ser bastante promissores quando 
comparados com aqueles de~\cite{Jaquet2019,Karch1999}. Isto é, os resultados morfométricos 
e o território vascular ocupado por cada árvore foi compatível com o fluxo alvo
desejado em cada simulação.

Destaca-se ainda que os modelos de árvores circulatórias gerados pelos algoritmos desenvolvidos satisfazem satisfatoriamente uma lei alométrica de sistemas biológicos apresentadas em ~\cite{West1997},  

Como trabalhos futuros, pretendem-se:
\begin{enumerate}
  \item Investigar propostas de paralelização/otimização de cada parte componente dos algoritmos;
  \item Analisar o comportamento dos algoritmos considerando variação no expoente de bifurcação e na viscosidade sanguínea;
  \item Aplicar o algoritmo em domínios de perfusão obtidos em casos clínicos reais e 
  verificar a viabilidade/qualidade dos modelos gerados para estudos hemodinâmicos. Através destes estudos, poderiam-se, por exemplo, obter informações das distribuições da pressão, fluxos e  de alguma substância (contraste) injetada na rede vascular do paciente que podem auxiliar a equipe médica em sua tomada de decisão.
\end{enumerate}