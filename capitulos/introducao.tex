\chapter{INTRODUÇÃO}

Neste capítulo, a contextualização da tese, os seus objetivos, as produções 
científicas oriundas até o momento desse trabalho e a estrutura do texto são 
apresentados.

\section{CONTEXTUALIZAÇÃO DA TESE}\label{sec:contextualizacao}

Uma parte importante no processo de estudar o sistema cardiovascular é a criação 
de modelos geométricos das árvores vasculares ao nível da circulação periférica 
devido à escassez de dados anatômicos dos leitos periféricos. Além disso, a 
atual resolução dos dispositivos médicos de aquisição de imagem (CT, MRI, etc.) 
não conseguem detectar vasos com diâmetro abaixo de uma fração de milímetros~\cite{Muller2008},
como por exemplo os que estão abaixo do nível epicárdio no coração~\cite{Jaquet2019}.
No momento apenas técnicas \textit{ex-vivo} podem ser usadas para 
segmentar estas regiões, como por exemplo a criomicrotomia~\cite{Goyal2013},
que basicamente consiste em endurecer os tecidos de interesse através de congelamento
e efetuar cortes sucessivos, delgados e uniformes. Desse modo, modelos computacionais
das árvores vasculares são muito úteis para simular e estudar a distribuição de fluxo 
sanguíneo até o nível da microvasculatura.

Esses modelos de árvores vasculares podem 
ser construídos usando, por exemplo, métodos baseados em leis fractais 
\cite{Van1989,Pelosi1987,Yang2013} 
ou métodos de otimização como o \textit{Constrained Constructive Optimization} 
(CCO)~\cite{Karch1999,Schreiner1993a,Schreiner1993b,Schreiner2006}. Dentre estas duas classes de métodos, 
o método de otimização é bastante aceito na literatura dado que uma hipótese razoável para os sistemas 
fisiológicos é que eles funcionam usando princípios de otimização~\cite{Kamiya1972,Murray1926,Thompson1992,Zamir1976}. 
Nestes sistemas, por exemplo, o sangue é um recurso limitado e ao mesmo tempo vital para o correto funcionamento 
do organismo. Parece condizente supor que a sua distribuição é otimizada de alguma forma.
 
Nos últimos anos, novas variantes do método CCO
foram propostas~\cite{Brito2017a,Brito2016,Brito2021,Georg2010,Jaquet2019,Meneses2017,Queiroz2013,Queiroz2018}
para torná-lo mais flexível e útil para construção de modelos com propriedades desejáveis, como por exemplo, 
o expoente da lei de bifurcação dependendo do 
nível de bifurcação~\cite{Meneses2016}. Ele também já foi 
utilizado como inspiração na construção de redes não vasculares, 
como por exemplo, as redes de Purkinje cardíacas~\cite{Ulysses2018}, que 
compõe o sistema de condução cardíaca responsável por gerar sinais elétricos e 
coordenar sua distribuição por todo o coração.

Tanto nos métodos formulados através de leis fractais quanto nos métodos 
baseados em princípios de otimização, as árvores vasculares
são consideradas basicamente como árvores binárias. Essas árvores binárias
são formadas por segmentos, representando pequenos trechos cilíndricos
de vasos, que estão conectados em bifurcações. Nas bifurcações, um segmento
pai está relacionado com seus dois segmentos filhos através de uma lei 
de potência que controla os raios envolvidos na bifurcação~\cite{Sherman1981}.

Em linhas gerais, dado um domínio de perfusão, os métodos baseados em leis fractais 
e o CCO possibilitam a construção de
uma única árvore vascular, seja ela para a distribuição de sangue e nutrientes 
(árvores arteriais) ou para a remoção do sangue e produtos metabólicos (árvores venosas). 
Entretanto, diversas regiões do corpo humano têm a característica de serem vascularizadas 
simultaneamente por mais de uma árvore vascular.

\section{OBJETIVOS}

O objetivo geral desta tese é desenvolver algoritmos que possibilitam 
a construção de mais de uma árvore vascular por vez em um mesmo domínio de perfusão. 
Já como objetivos específicos, destacam-se os seguintes:
\begin{itemize}
  \item desenvolver e implementar algoritmos baseados no método CCO para a construção
  de florestas de árvores arteriais dentro do domínio de perfusão;
  \item investigar propriedades morfométricas dos modelos gerados e leis alométricas.
\end{itemize}

\section{PRODUÇÕES CIENTÍFICAS}

Os resultados deste trabalho geraram as seguintes produções científicas:

\begin{itemize}
 \item Queiroz, R. A. B., Aquino, L. C. M. Automatic Construction of Vascular Arteriovenous 
 Tree Geometric Model. Proceeding Series of the Brazilian Society of 
 Computational and Applied Mathematics. vol. 6, n. 2. 2018.
 \item Aquino, L. C. M. de, Queiroz, R. A. B., Rocha, B. M. Construção automática de múltiplas
 árvores circulatórias com controle de invasão de território. 
 \emph{Trends in Computational and Applied Mathematics}. Em processo de revisão na revista.
\end{itemize}

\section{ORGANIZAÇÃO DA TESE}

Esta tese é dividida em sete capítulos, sendo o primeiro formado por essa introdução e os demais conforme seguem:

\begin{itemize}
  \item \textbf{Capítulo 2: MODELOS COMPUTACIONAIS DE ÁRVORES CIRCULATÓRIAS}
  
  Neste capítulo é detalhado o método CCO, bem como abordagens baseadas nele
  para construir múltiplas árvores dentro de um domínio de perfusão.
  
  \item \textbf{Capítulo 3: ALGORITMOS PROPOSTOS PARA CONSTRUÇÃO DE FLORESTA DE ÁRVORES ARTERIAIS}
  
  Neste capítulo é apresentado um algoritmo para a construção de uma floresta 
  de árvores arteriais considerando um controle de invasão. Além disso, também
  é apresentado um algoritmo que constrói uma floresta de árvores em dois estágios usando o diagrama de Voronoi.
  
  \item \textbf{Capítulo 4: RESULTADOS ALOMÉTRICOS E MORFOMÉTRICOS}
  
  Neste capítulo são exibidos e discutidos os resultados 
  alométricos e morfométricos obtidos usando os algoritmos descritos no Capítulo 3. Por fim, o capítulo 
  encerra com a discussão sobre os resultados obtidos nas simulações.

  \item \textbf{Capítulo 5: CONCLUSÕES E TRABALHOS FUTUROS}
  
  Neste capítulo são expostas as conclusões gerais desse trabalho,
  bem como são indicadas algumas propostas de trabalhos futuros.
  
  \item \textbf{APÊNDICE A: Notas sobre implementação computacional}
  
  Neste apêndice são apresentados alguns comentários sobre a implementação dos algoritmos
  propostos nesta tese.
\end{itemize}
